\chapter{Discussion}

% In the previous chapters, we have described a new language for representing SME
% networks, SMEIL and how its implementation fits into the greater whole
% \todo{rewrite}. The questions that are left to answer now are, what have we
% achieved by doing so? Is SMEIL suitable for the current SME ecosystem, and
% finally,

We started working on SMEIL with the intention of creating an intermediate
representation for use with existing SME implementations. The resulting language
covers a wider scope and has a larger number of use cases than was originally
intended. In this chapter, we discuss uses for SMEIL and how it relates to
current SME implementations.

% \section{Completeness of SMEIL in relation to SME}
% In this section, we try to provide some insight into whether the SMEIL provides
% a {\itshape complete} representation of the SME model. Thus, a complete SME
% implementation is \begin{inparaenum}]a)] \item able to construct networks from 
% \end{inparaenum}

% A complete SME implementation implements all the key elements of the SME model
% including
% \begin{itemize}
% \item Independent processes
% \item Buses
% \item Synchronous communications
% \item Ability to create compositions of networks.
% \end{itemize}
% as SMEIL features all of these, we believe it to be able to provide 

% As we have shown in several examples. 

% Before SMEIL is able to be used 
% A shortcoming of SMEIL in its current state is that we do not yet support all of
% the features of the C\# SME implementation, however, doing so was not within the
% scope of the thesis in the first place.


\section{Usability of SMEIL}
As a language for representing SME networks SMEIL is complete in the sense that
it implements the elements and semantics required by the SME model. We have also
shown a number of actual implementations of SME networks showing the practical
use of SMEIL. Thus, SMEIL is capable of representing the SME models which are
currently implemented in other languages.

% This is shown here through the numerous examples that we have been able to
% implement.
% An issue which we have not discussed is how well SMEIL scales for
% larger networks since we have only looked at small examples. For this, we argue
% that the scalability of SMEIL is similar to that of C\# SME which has previously
% been used to create large-scale SME implementations. This is because the primary
% structural components of the C\# SME implementations maps to SMEIL in a
% straight-forward manner.


We have previously argued for the potential benefits of using SMEIL in favor of
general-purpose for writing hardware-targeted SME models. However, there are
also several disadvantages related to implementing a new and distinct
language. One particular disadvantage is the abandonment of development
environments, debuggers and other tooling that comes with an established
general-purpose language. For SMEIL to offset these disadvantages, an additional
development effort is required to replace this tooling. On the other hand,
debuggers for conventional programming languages are not well suited for
debugging concurrent models, such as SME. Therefore, introducing a new language,
may create an avenue for creating new tooling which are better suited for their
purpose.

% On the other hand,
% in relation to debuggers it could be argued that languages for constructing
% concurrent models, such as SMEIL, require a different kind of debuggers than
% traditional sequential languages.


% Another aspect which is outside the scope of the thesis is actually implementing
% the generated VHDL code on hardware. However, it is still an important aspect to
% discuss since the primary purpose of writing SME models is to, as previously
% mentioned, eventually create a hardware description....


\subsection{Using SMEIL as an IL}
SMEIL is usable as an intermediate language for SME models written in a wide
range of different general-purpose languages. The static structural definitions
and the static type system of SMEIL forces a normalized representation of SME
networks regardless of which source language the SMEIL code was generated
from. Thus, when generating SMEIL from a dynamically typed source, it is the
responsibility of the generator to ensure that the appropriate types are
assigned. In PySME, for example, this problem was solved by allowing
type-annotations to be written in comments. The result of this, is that a SMEIL
program should not inherit any particular traits of the code that it was
generated from. Therefore, SMEIL will allow combining SME networks that was
originally written in different source languages.

We have already shown an example of how SMEIL can function as an intermediate
language for the PySME SME implementation. However, the PySME library is
restricted in terms of functionality compared to the C\# SME library which has
been used as the base for all of the previously referenced SME models
implemented on hardware. Compared to C\# SME, SMEIL and \libsme{} still lacks a
number of essential features. Achieving feature-parity with ``state-of-the-art''
SME libraries was considered outside the scope of this thesis due to the longer
period of time that these libraries has been under development. However, it
means that more work is required in order to make SMEIL usable as an IL for C\#
SME.

% \section{Relation with other HDLs}
% Discuss how SME and specifically SMEIL relates to other similar ``simple''
% hardware description languages. \todo{Should this be here?}

% \section{Related co-simulation approaches}
% \todo{Should this be here}

% \section{Comparison with ``state-of-the-art'' SME}
% Write about: Internal buses: We have language support based on inference from
% usage, but not implemented. Arrays as buses

\section{Co-Simulation and observation based typing}
One of the strongest features of SMEIL is the co-simulation interface which
allows it to seamlessly interact with SME networks written in other
languages. It is the cornerstone of enabling SMEIL to be used both as an
intermediate language and as a primary implementation language for SME models in
a capable manner. For the intermediate-language use case it helps simplifying
the implementation of SME libraries using it. For example by offloading the
generation of trace-files. For use as a primary implementation language, it
allows direct testing of designs without having to execute generated
code. Furthermore, the co-simulation interface is central in enabling a tightly
integrated implementation of the system for observation-based typing.

Co-simulation is a frequently used technique for verifying hardware designs (see
e.g. CoCoTB referenced in \Cref{sec:retwork}), however, the presented SME-based
solution is unique in that the same model is used on both sides of the
simulation. As one particular advantage, the compositionality of SME means that
a process may be moved seamlessly between both sides of the Co-Simulation.  We
have only shown an implementation of the API for Python here but we believe that
extending other SME implementations with support for the \libsme co-simulation
API will be possible. This is the case since the API is kept ``as close to C''
as possible. Hence, any language featuring a capable C interface should be able
to use it.

% Due to the dynamic nature of Python it lends itself well for performing
% such...

Our approach for determining types based on observed input is, to the best of
our knowledge, also not implemented in any existing HDLs. We are aware that
there may be good reasons for this, due to the, as previously mentioned,
inadequate safety provided by this approach. However, based on our limited
testing, the productivity advantage of not having to place precise type bounds
on all internal variables does seem to be real. Furthermore, in some cases, such
as quickly prototyping a design on a FPGA, safety may not be the primary desired
features. Thus, if combined with static analysis to prove that the
observationally derived type bounds really are safe, it may turn out to be a
powerful approach.

\section{Target Language Support}
We have currently only implemented a code generation backend for VHDL. We
focused on supporting VHDL since the primary intended target for SME models are
hardware designs. For this thesis, we chose not to implement support for
additional code generation backends and instead making the core infrastructure
related to SMEIL as complete as possible. However, having support for generating
additional languages is desirable in several circumstances. Having, for example,
C++ support will, as shown in \cite{skovhede2017c++}, allow for faster
simulation of SME networks and allow SME networks to be used as independent
libraries with other languages.

Supporting other HDLs, such as Verilog, is also desirable since it would allow
SME models to interface with hardware designs implemented in HDLs other than
VHDL. Adding such support to SMEIL is quite feasible since most other ``new''
HDLs (we do a small survey in \Cref{sec:retwork}) support code generation for at
least VHDL and Verilog. This shows that hardware designs which are implemented
in VHDL can also be represented using Verilog.

The current \libsme{} implementation is written with support for multiple target
languages in mind. This is done, by confining all VHDL-related details to the
VHDL code generation backend. All other phases of the compiler are kept as
general as possible. Thus, we do not foresee any problems related to adding
additional code-generation backends for SMEIL.

We have not assessed the practicality of transforming SMEIL to languages
following other paradigms, such as functional languages or alternative hardware
description languages. So far, however, there has not been a need for SME models
to be able to target such languages.


% However, compiling SMEIL to other languages, such as C++ ) is also desirable. Furthermore, \libsme is written with
% support for multiple target languages in mind, and the infrastructure to add
% another target language is in place, but unfortunately time constraints did not
% allow us to do it. Specifically in relation to other HDLs such as Verilog. Most
% new high-level HDLs is able to generate at lest VHDL and Verilog. SME is
% currently only translatable to VHDL, however, adding support for Verilog is
% entirely possible. To support this claim, we point
% to \begin{inparaenum}[a)]\item The large number of tools (e.g. \todo{cite})
%   )already supporting both languages and \item numerous \end{inparaenum}
% \todo{cite} \todo{Why do we think its simple to add another language?}

%%% Local Variables:
%%% mode: latex
%%% TeX-command-extra-options: "-enable-write18"
%%% TeX-master: "../master"
%%% End:
