\section{Motivations for Custom Hardware}
% \todo{Write this}. Give a brief overview of a hardware design workflow and give
% an overview of especially VHDL as this is discussed here repeatedly.

% %Basic hardware architecture -- gate level

% %FPGAS -- Reconfigurability, LUTs, BRAM, m.m.

% %Programming models -- HDLs (VHDL), ..

% \todo{This needs a shitload of references}

% This section is an attempt to give a more in-depth motivation of why 

%\subsection{A brief history}
Digital circuits are fundamentally a collection of logic gates which are
connected in a specific configuration to perform a particular purpose. In
Integrated Circuits (ICs), this configuration is hard-wired -- etched into
silicon using a lithographic process. An example of a hard-wired IC is the
common Central Processing Unit (CPU). CPUs are highly versatile devices, capable
of computing anything computable. While this versatility is their main
advantage, it also mean that they excel at nothing.
% \todo{mention von
%   neumann}\footnote{To keep this short, we don't discuss the widely available
%   Single Instruction Multiple Data (SIMD) instruction sets such as 3Dnow!, MMX,
%   SSE and AVX. These implement data-level parallelism in a restricted
%   manner}.
Since the advent of the first microprocessor, almost 5 decades ago, this problem
has been widely recognized. Therefore, a steadily increasing amount of
special-purpose hardware is being added in order to relieve the CPU of common
and computationally intensive task. An early, and highly successful, example of
this is the introduction of co-processors for performing floating-point
calculations. As personal computers were increasingly being used for tasks
relying heavily on floating-point arithmetic, emulating this in software
increasingly became a limiting factor for performance. To solve this,
specialized hardware units, known as Floating Point Units (FPU) were added to
significantly speed up floating point calculations compared to what was possible
using software emulation. Numerous similar examples exists, for example the
AES-NI instruction set built into recent CPUs, providing access to specialized
circuitry for performing encryption and decryption using the ubiquitous Advanced
Encryption Standard (AES) algorithm.

The current use of specialized hardware only scratch the surface of applications
for which it would be beneficial. For example, by offering a significantly
improved performance-per-watt ratio~\cite{fowers2012performance}. This is
especially important in the light of the
ever-increasing~\cite{avgerinou2017trends} power consumption of data centers
which counter global efforts to decrease emissions. In an ideal world, everyone
would have cheap and easy access to creating hardware specialized for their
particular application. Unfortunately such hardware, known as Application
Specific Integrated Circuits (ASICs), is extremely expensive and time-consuming
to design and put into production and a very large number of units has to be
ordered before achieving a reasonable cost-per-unit. Also, mistakes are
expensive since once the hardware is made it can never be changed.

% Another well known example of special-purpose hardware are Graphics Processing
% Units (GPU) introduced to, as the name suggests, accelerate graphics in
% computers. Therefore, GPUs are optimized for workloads that are ubiquitous in
% graphics processing such as bulk operations on arrays. Therefore, GPUs are
% massively parallel machines, capable of performing a single operation, in
% parallel on a large set of data.


% In 1965 Gordon Moore famously stated that the number of transistors on a single
% IC would double every year. This is commonly known as Moore's Law and has ensured
% a steady increase in performance ever since. Today, Moore's law i

\subsection{Field Programmable Gate Arrays (FPGA)}
Reprogrammable computing, of which Field Programmable Gate Arrays (FPGAs) are
the only prominent example, offer an attractive compromise between ASICs and
more general devices such as CPUs or GPGPUs. While not nearly as good as
ASICs~\cite{kuon2007measuring}, they can provide significant improvements in
both performance and power-consumption while also offering a significantly lower
cost.

FPGAs are ICs which allow their circuits to be changed after manufacture (hence,
they are programmable in the field) and can therefore be reconfigured to fulfill
any purpose. Of course, the circuitry on the chip cannot be physically changed,
so FPGAs consists of an array of logic blocks which has configurable
interconnects. The reprogramming of an FPGA happens by reconfiguring switches
which determines the signal path. Since the individual components on an FPGA are
fine-grained, they can be reconfigured for any computational purpose by rewiring
signal paths.

\subsection{Programming FPGAs}
As briefly mentioned in the introduction, FPGAs are usually programmed using
Hardware Description Languages (HDLs) such as VHDL or Verilog. We will focus on
VHDL here, since that is the current target of SMEIL code generation.

VHDL was originally developed in the late 1980's as a formal specification
language for documenting hardware design s

\todo{finish}
%VHDL was originally developed as 

%%% Local Variables:
%%% mode: latex
%%% TeX-master: "../master"
%%% End:
