\chapter{Analysis}

\section{Definition Structure}
Early descriptions of \gls{sme} defined networks in a rather rigidly, by
requiring all buses, process instantiations and links between them to be placed
within a network definition. Building \gls{sme} networks in this manner proved
limiting and confusing as it was easy to loose track of how buses within a
network was connected. The C\# implementation of \gls{sme} loosened this
restriction by defining buses directly within processes. Connections between
processes was then created simply by referring to a bus within another
process. This proved, in practice, to be much more programmer friendly as it
made the connections between processes easier to reason about. The limiting
factor of this design was the bijective relationship between process
declarations and their instances. Overcoming this limitation in practice
required the inelegant use of plain-text templates for creating multiple
duplicate process and network definitions. To remove this limitation, support
for ``Scopes'' was added to C\# \gls{sme}. Scopes a

Each network definition creates an enclosed scope around the processes it
instantiates. Therefore, processes from other networks may only connect to buses
which are exposed through another network definition. 

\section{Network Instantiation}
Instantiation of process networks defined in \gls{smeil} is a topic which
deserved particular attention. In \gls{sme} libraries for general-purpose
languages, the networking structures are explicitly defined and network
instantiations happen naturally when the code is execution. The actual
instantiation process here is, as we might say, handled by the programming
language that the networks are instantiated within. For \gls{smeil} we aren't as
lucky and a method needed to be developed. Valid \gls{sme} networks are
representable as a connected directed acyclic graph. Acyclicity is a required
property as we would otherwise end up with infinite instantiation loops. The
buses of \gls{sme} networks may, however, form loops. Once the graph has been
created and we have shown the graph to be acyclic, we topologically sort the
nodes and start instantiating the network from the first node in the sorted
nodelist.  TODO: Show examples and write some more about this. The method of
instantiation for \gls{sme} networks is summarized as follows:
% \begin{itemize}
% \item Construct a graph from \gls{sme} network definitions
% \item Show that graph is acyclic
% \item Perform a topological sort of nodes in the graph. Instantiate the network
%   from the first entry of this graph. Alternatively, simply start from the node
%   with in-degree of 0. (TODO: Is this valid?)
% \end{itemize}

TODO: Clarify relationship between instantiation graph and connection graph.

%%% Local Variables:
%%% mode: latex
%%% TeX-master: "../master"
%%% TeX-command-extra-options: "-enable-write18"
%%% End:
