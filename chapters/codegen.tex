\chapter{Code Generation}
\label{sec:codegen}
SMEIL compiles to clean and readable VHDL code which is amenable to manual
modifications, should this be desired. The code may be executed using a VHDL
simulator or passed to FPGA vendor tools for synthesis and subsequent
hardware-implementation. The generated code is a cycle-accurate representation
of the original SMEIL network.
\todo{Make more detailed or don't use a separate chapter.?}

\section{Naming in VHDL}
A key issue is how to handle the transformation of names when moving from
transforming an SMEIL program to VHDL. Since an entity in an SMEIL program
translates directly to a VHDL entity, we need to make sure that names in the
original SMEIL program remains unique
\todo{Show example from EWMA top-level entity}

\section{Code transformations}
%In the f

The fundamental structure of the SMEIL code is preserved in the generated
VHDL. One VHDL entity is generated per SMEIL entity and the body a SME process
is transformed to a sequential process in VHDL. For each of these processes, we
also generate code for performing an asynchronous reset of all variables and
outgoing signals. The naming hierarchy of the original SMEIL is preserved,
making it easy to identify from where a particular section of the VHDL code was
generated.

For verifying the generated VHDL code, a test-bench is also generated. The
test-bench is connected to the {\ttfamily exposed} buses of the SMEIL program. The
CSV-trace file containing the values recorded during simulation is used by the
VHDL test bench to drive inputs and verify outputs.

Alongside the generated code, a {\ttfamily Makefile} is generated for building
and testing the VHDL code using the GHDL simulator.

Integer types of SMEIL are represented in VHDL using the types provided by the
standard {\ttfamily ieee.numeric\_std} package. This package provides functions for
performing signed and unsigned integer arithmetic with logic-vectors. For
example, the types {\ttfamily i4} and {\ttfamily u12} are represented as {\ttfamily signed (3
  downto 0)} and {\ttfamily unsigned (11 downto 0)} respectively. Arrays require the
creation of a new {\ttfamily type} in VHDL. A {\ttfamily type} declaring a 10-element array
of 5-bit signed integers ({\ttfamily [10]i5} in SMEIL) is represented in VHDL as
\begin{lstlisting}[language=vhdl]
type \[10]i5\ is array (0 to 9) of unsigned (4 downto 0);
\end{lstlisting}
These type declarations are stored in a separate package, {\ttfamily sme\_types.vhdl},
which is shared between all entities of the design to avoid cluttering the
generated code with duplicated declarations. SMEIL booleans are represented
using the VHDL type {\ttfamily boolean}. As an alternative to this, a single {\ttfamily
  std\_logic} type is commonly used. This type represents a wire in the hardware
and is, therefore, able to have other states than just true or false. This may
be useful in some circumstances.

The actual code is generated using the {\ttfamily language-vhdl-quote}
library~\cite{language-vhdl-quote} which provides
quasiquoters~\cite{mainland2007s} for building VHDL ASTs using the concrete VHDL
syntax. The major advantage of this approach is that it minimizes the chance of
generating syntactically invalid VHDL since syntax errors are caught during
compilation of \libsme{}. Furthermore, a complete VHDL AST is constructed
containing the contents of each generated VHDL file. This AST is then
pretty-printed, yeilding consistently formatted code which is difficult to
achieve using more common techniques using string-based templates.


%%% Local Variables:
%%% mode: latex
%%% TeX-master: "../master"
%%% End:
