\chapter{Co-simulation}
\label{sec:cosim}

% Without a method for providing external input to SME networks  would not be very
% useful.
The usefulness of SMEIL would be limited without a way to provide external
interactions with SME networks written in SMEIL. The simplicity of SMEIL can be
attributed to its narrow scope: it is only intended for writing hardware models
and not for writing test benches. For this, the full power of a general-purpose
language is needed as the test-code can be written without hardware-related
considerations and using all available libraries. For example, a test bench may
read an image from disk or visualize the results of a simulation. Extending
SMEIL to be able to perform such tasks is a substantial undertaking that does
not further its primary purpose as a hardware-modeling
language. Co-simulation~\cite{schloegl2015towards} is the process of two
separate entities (in this case two SME networks) which communicates through
channels transparently established by the SME libraries.

\section{Co-simulation with SMEIL}
For performing co-simulation with SMEIL, we expose a C API from \libsme{}. The
API is intended to be implemented by SME libraries for general-purpose
languages.

There are three aspects to the API: Firstly it provides a way to enumerate the
buses exposed from a SMEIL model. Secondly, it provides calls for reading to-
and writing from bus channels and driving forward the simulation. Finally, it
offers calls for ordering the production of output, such as VHDL code
generation.

A noticeable feature of the API is its support for arbitrary-length
integers. This means that the bit-width of buses used by simulated models are
not restricted by the register-width of the CPU of the underlying host. A
hardware model may use values of any bit-width and the model should, therefore,
not inherit the limitations of the platform that it is simulated on.

% Support for
% simulating designs using integers larger than the register-width of the host CPU
% is desired since it allows them to be used in hardware models. \todo{fix}

When the co-simulation API is used \libsme{} operates as a ``puppet'', being
controlled by the calling program (the ``puppeteer''). Only buses declared with
the \texttt{exposed} (\Cref{sec:langref}) modifier in SMEIL are accessible
through the \libsme{} API. The calling library drives forward the simulation by
calling a function for ticking the clock and for reading and writing from/to the
exposed buses of a SMEIL program. In the following section, we give a detailed
introduction to the API.  For brevity, we refer to an SME library implementing
the API as the ``client'' in the following.

% \begin{figure}
%   \resizebox{.7\linewidth}{!}{
%     \begin{tikzpicture}[>=latex]
%       \coordinate (A) at (2,5);
%       \coordinate (B) at (2,0);
%       \coordinate (C) at (6,5);
%       \coordinate (D) at (6,0);

%       \draw[thick] (A)--(B) (C)--(D);

%       \draw (A) node[above]{libsme};
%       \draw (C) node[above]{Client};

%       \coordinate (E) at ($(A)!.25!(B)$);
%       \coordinate (F) at ($(C)!.1!(D)$);
%       \draw[->] (F) -- (E);
%     \end{tikzpicture}}
% \end{figure}


The approach presented here is conceptually similar to the Verilog Procedural
Interface (VPI)~\cite{dawson1996verilog} which is used for interfacing with
Verilog and VHDL simulators. However, a big advantage of the SMEIL approach is
that SME is used on both sides of the co-simulation. Hence, both the functional
and verification parts of the network act as a single unified entity. Thus, the
programmer does not need to consider integrating different abstract interfaces.


\section{API reference}
This section documents the public API of the co-simulation interface of
\libsme{}. Towards the end of the section, we give a short overview of how the
API is used.

\subsection{Exported data structures}
\begin{lstlisting}[language=c,multicols=2]
typedef enum Type {
  SME_INT,
  SME_UINT,
  SME_FLOAT,
  SME_DOUBLE,
  SME_BOOL
} Type;

typedef struct SMEInt {
  int len;
  int alloc_size;
  int negative;
  char* num;
} SMEInt;

typedef struct Value {
  Type type;
  union {
    bool boolean;
    SMEInt* integer;
    double f64;
    float f32;
  } value;
} Value;

typedef struct ChannelRef {
  char* bus_name;
  char* chan_name;
  Type type;
  Value* read_ptr;
  Value* write_ptr;
} ChannelRef;

typedef struct BusMap {
  int len;
  ChannelRef** chans;
} BusMap;
\end{lstlisting}

The data structures listed above constitute the primary interface for exchanging
data with \libsme{}.

\subsection{Public API}
\begin{description}
\item[\texttt{SmeCtx* sme\_init()}]\hfill\\
   Initializes and returns the SME library context.
 \item[\texttt{bool sme_open_file(SmeCtx* ctx, const char* file, int argv,
     char** argc)}]\hfill Loads an SMEIL file, while applying the supplied
   arguments to \libsme{}.
   \item[\texttt{bool sme_has_failed(SmeCtx* ctx)}]\hfill\\
     Returns true if an operation within the \libsme{} library failed.
 \item[\texttt{char* sme_get_error_buffer(SmeCtx* ctx)}]\hfill\\
   Returns a string containing the error message emitted by \libsme{}. The memory
   pointed to may not be freed except by calling the \texttt{sme_free} function.
 \item[\texttt{void sme_free(SmeCtx* ctx)}]\hfill\\
   Frees the SME library context and related resources.
 \item[\texttt{bool sme_tick(SmeCtx* ctx)}]\hfill\\
   Ticks the clock of an SME simulation synchronously. When this function
   returns, all processes defined within \libsme{} will have run and written to
   their buses.
 \item[\texttt{bool sme_finalize(SmeCtx* ctx)}]\hfill\\
   Finalizes a simulation and dumps the recorded trace file (if any) to the file
   system. This function should always be called following the final call to
   \texttt{sme_tick}.
 \item[\texttt{bool sme_propagate(SmeCtx* ctx)}]\hfill\\
   Propagates the values of both internal and external facing buses defined in
   \libsme{}. Run this function before the clock is advanced (by calling
   \texttt{sme_tick}) in the simulation loop and it should be run together with
   any bus propagation that needs to be performed by the calling code. When this
   function returns, the values of all buses defined within \libsme{} have been
   propagated.
 \item[\texttt{void sme_integer_resize(SMEInt* num, int len)}]\hfill\\
   When manipulating values of type {\tt SMEInt} (arbitrary-size integers) the
   \texttt{sme_resize_integer} function will make sure that the memory pointed
   to by {\tt Value.value} is large enough to hold the number that you intend to
   store. The function takes a pointer to the {\tt SMEInt} structure and a
   parameter len which is the number of bytes needed to store the number as
   base-256. This function must be called before every direct manipulation of
   {\tt SMEInt.num}. For a safer interface, see \texttt{sme_integer_store}.
 \item[\texttt{void sme_integer_store(SMEInt* num, int len, const char val[])}]\hfill\\
   Stores the base-256 representation of an integer in an \texttt{SMEInt}.
 \item[\texttt{void sme_set_sign(SMEInt* num, int sign)}]\hfill\\
   Sets the sign of an \texttt{SMEInt}. Possible values for \texttt{sign} are 0
   meaning the number is positive and 1 for a negative value.
 \item[\texttt{BusMap* sme_get_busmap(SmeCtx* ctx)}]\hfill\\
   Returns a pointer to a {\tt BusMap} structure containing the exposed buses of
   the SME network. This function is intended to be used by implementers of
   \libsme{} to generate internal representations of their SME buses. It is the
   callers responsibility to free the memory returned by the function by calling
   \texttt{sme_free_busmap}.
\item[\texttt{void sme_free_busmap(BusMap* bm)}]\hfill\\
   Frees a BusMap structure allocated by \texttt{sme_get_busmap}.
\end{description}

\subsection{Using the API}
The initial interaction with the API happens through a call to the {\tt
  sme_init} function. The {\tt SmeCtx} pointer returned is an opaque reference
to the SME library context which must be included in every future interaction
with the library. The next step is to load a SMEIL program by calling {\tt
  sme_open_file}. Then, the implementing client must retrieve a {\it busmap}
from \libsme{} in order to learn how to access the {\tt exposed} buses of the
SMEIL program. This is done by calling the {\tt sme_get_busmap} function. As
seen, it returns a pointer to a {\tt BusMap} struct which then again points to
an array of {\tt ChannelRef} structs. Each of these contains an individual bus
channel. In addition to the type of the channel, a {\tt ChannelRef} also
contains two pointers to {\tt Value} structs. One for the reading- and
writing-end of the channel respectively. Reading to- and writing from the bus
channel is done by directly accessing these {\tt Value}s. A {\tt Value} consists
of the actual value plus its type. The only value which is not simply
represented using a native C type is {\tt SMEInt} which, as seen, contains a
{\tt char*}. This is a pointer to the memory used for storing the individual
bytes of a base-256 integer. Before this memory is written, the {\tt
  sme_integer_resize} function must be called to ensure that the memory is large
enough to contain the number that the client intends to store.

After each of these calls, the {\tt sme_has_failed} must be called in order to
check if the previous operation resulted in an error. If the function returns
{\tt true}, the {\tt sme_get_error_buffer} function returns a pointer to a
buffer containing a human-readable description of the error which occurred.

In the main simulation loop of the implementing client library the following
calls to the API are usually made: In the bus propagation phase, the client must
propagate its own buses and then call the {\tt sme_propagate} function to
propagate the buses defined the SMEIL program. Following bus propagation, the
client library must run its own processes and the processes defined in the SMEIL
program. The processes defined in the SMEIL program are run by calling the {\tt
  sme_tick} function.

When the simulation is complete and the desired number of cycles has been
performed, before freeing memory, a final call to {\tt sme_finalize} must be
issued in order to dump the recorded trace file to disk.

The intention of this API is that it should be as general as possible such that
it can be implemented by any language providing an adequate Foreign Function
Interface (FFI) for C APIs.

\section{Co-simulation using PySME}

\begin{figure}
  \centerfloat
  \begin{subfigure}[t]{0.40\paperwidth}
\begin{lstlisting}[language=smeil]
proc addone(in inbus, const val)
    exposed bus addout {
       val: i32;
    };
{
    addout.val = inbus.val + val;
}

network addone_net() {
    exposed bus idout {
        valid: bool;
        val: i32;
    };

    instance addone_inst of
        addone(idout, 1);
}
\end{lstlisting}
    
\subcaption{The SMEIL code in \texttt{addone.sme}.}
  \end{subfigure}
  \begin{subfigure}[t]{0.40\paperwidth}
    \begin{lstlisting}[language=python]
from sme import *

class Id(SimulationProcess):
  def setup(self, ins, outs, result):
    self.map_outs(outs, "out")
    self.map_ins(ins, "inp")

  def run(self):
    result[0] = self.out["val"]
    self.out["val"] = self.inp["val"]

@extends("addone.sme", ["-t", "trace.csv"])
class AddOne(Network):
  def wire(self, result):
    plus_out = ExternalBus("addone_inst.addout")
    id_out = ExternalBus("idout")
    p = Id("Id", [plus_out],
      [id_out], result)
    self.add(plus_out)
    self.add(id_out)
    self.add(p)

if __name__ == "__main__":
  sme = SME()
  result = [0]
  sme.network = AddOne("", "AddOne",
             result)
  sme.network.clock(100)
  print("Final result was ", result[0])
  \end{lstlisting}
  \subcaption{The corresponding Python code.}
  \end{subfigure}

  \caption{Example code showing the an interaction between SMEIL (left) and PySME
    (right).}
\label{fig:smeilpy}
\end{figure}

In order to show a client implementation of the API described above, we have
extended the PySME library~\cite{pysme} with support for co-simulation enabling
seamless interaction between SME networks written in Python and SMEIL. In
practice, extending the PySME library was straightforward and required less than
a day of implementation work by a person with expert knowledge of both
code-bases. We expect that a similar effort is required to extend other SME
implementations (such as C\# SME and C++ SME).

As an example, we revisit the {\sc addone} example introduced in \Cref{sec:smeilex},
this time implementing one half of it in Python as seen in
\Cref{fig:smeilpy}. The \texttt{@extends} decorator is all that is needed to
make the buses exposed from the SMEIL network available for the Python
program. Behind the scenes, \libsme{} is loaded and the \texttt{addone.sme} file
is parsed, typechecked and the \libsme{} SMEIL simulator is initialized. We gave
a detailed description of this interaction in the previous section. A
SMEIL-defined bus is referenced from PySME by instantiating an
\texttt{ExternalBus}, providing the name of a bus defined in the SMEIL program
as its parameter. The semantics of an {\ttfamily ExternalBus} is identical to
those of a bus defined within Python. Any SMEIL type except strings and arrays
may be passed along a bus. Strings play a very limited role in SMEIL and are
therefore unlikely to be supported. Arrays, on the other hand, are desirable to
include in future extensions. Integers are encoded in base-256 as a sequence of
bytes, allowing arbitrarily-sized integers to be used between co-simulated
entities.

When this program is run, the PySME library calls the \libsme{} library for
every cycle to stepwise progress the simulation. During the simulation \libsme{}
may, if asked to do so, record a trace of the communication taking place over
the buses to a file. This trace file is then later used as the data source for
the VHDL test bench which is used to verify the generated VHDL code.

This is a highly flexible model as co-simulation is enabled with minimal
intrusion on existing PySME code. For example, should \libsme{} be extended with
a high-performance simulation backend for SMEIL, existing programs can take
advantage of this without modifications. The implementation of \libsme{} may
even be replaced entirely, as long as the current API is
maintained. Furthermore, it also facilitates an incremental design strategy,
where a Python prototype can gradually be rewritten in SMEIL.

Notice in the PySME code of \Cref{fig:smeilpy} that the {\tt addout} bus is
referenced through its instance name {\tt addone_inst.addout} and the {\tt
  idout} bus is referenced directly by its name. This is because the latter bus
is declared directly inside the top-level entity while the former is declared
within the {\tt addone} process which is instantiated as {\tt addone_inst}.
This naming scheme is used to ensure that exposed buses have unique names.

We show more examples of using co-simulation for testing SMEIL networks in
\Cref{eval}.

% \section{Runtime representation}
% In order to perform a simulation of  SMEIL network, we transform it to a
% suitable runtime representation. 


\section{Typing networks through simulation}
\label{sec:typing}
% \todo{Why do we only have this restricted notion of dynamic typing. Why no
% completely variable types?}
As described in \Cref{sec:typesys} SMEIL supports integers of both constrained
and unconstrained bit-widths.  In order to translate SMEIL to a hardware
description, we require that all types in the program are constrained to a
specific bit-width.  In a hardware description, we need to statically specify
the number of bits required by each value and therefore, arbitrary-width
integers are not representable.  However, it is often difficult to decide the
optimal bit-width of a value in advance. In particular, this applies to internal
variables whose values are derived from external inputs. To address this,
\libsme{} provides a method for re-typing a SMEIL program based on values
observed during simulation.

When this feature is enabled, the maximum absolute value assigned to a variable
is stored alongside its current value. Whenever the variable is assigned, the
new value is compared to the current maximum which is then updated as needed.

% As described in \Cref{sec:typesys}, the only unsized types which exists in SMEIL
% are signed and unsigned integers. This raise the question: Why not have an
% {\ttfamily any}-type which could unify with any other type? A related form of
% the question is: Why distinguish between signed and unsigned unsized integer
% when the signed type is the superset of the two

%\todo{Describe implementation details}
%\todo{We promise to mention ranges earlier}
When the simulation is concluded, the observed value ranges are converted to
SMEIL types large enough to hold the range.  For example, in the program shown
in \Cref{fig:non-violated}, we observed that the bus channel {\tt b.chan} was
assigned values between 0 and 29 during simulation. Therefore, the bus channel
is assigned the type {\tt i6} as we need 6-bits to hold the value 29 in a signed
integer.

The types and observed ranges are spliced into the SMEIL AST and the re-typed
program is then passed through the type checker. This ensures that constraints
originating from fixed-size types in the original program are not violated. This
process is illustrated in \Cref{fig:simtyping} which shows how observationally
derived types are spliced into an existing program. \Cref{fig:violated} shows
the violation of an existing constraint in the program. Since the value
{\ttfamily c} has the fixed-sized type {\ttfamily i10}, the program will no
longer be valid if {\ttfamily b.chan} observes values that are 16-bit long. A
configuration flag {\ttfamily -{}-no-strict-type-bounds} overrides this behavior
by considering all types as unconstrained (i.e., {\ttfamily i10} is considered
identical to {\ttfamily int}).


\begin{widefigure}
  \centering
  \begin{subfigure}[t]{0.30\linewidth}
    %\centering
\begin{lstlisting}[language=smeil]
proc A ()
 bus b {
  chan: ?{\bfseries\underline{int}}?;
 };
 var c: i10;
{
 c = b.chan;
}
\end{lstlisting}
    \caption{Unconstrained types.}
    \label{fig:oktype}
  \end{subfigure}
  \begin{subfigure}[t]{0.30\linewidth}
    %\centering
    \begin{lstlisting}[language=smeil]
proc A ()
 bus b {
  chan: ?{\bfseries\underline{i6 range 0 to 29}}?;
 };
 var c: i10;
{
 c = b.chan;
}
    \end{lstlisting}
    \caption{Valid.}
    \label{fig:non-violated}
  \end{subfigure}
  \begin{subfigure}[t]{0.30\linewidth}
    %\centering
    \begin{lstlisting}[language=smeil]
proc A ()
 bus b {
  chan: ?{\bfseries\underline{i16 range 0 to 30717}}?;
 };
 var c: i10;
{
 c = b.chan;
}
\end{lstlisting}
    \caption{Invalid.}
    \label{fig:violated}
  \end{subfigure}
  \caption{Shows a process entering the simulator with an unconstrained type (a)
    and examples of two possible resulting programs (b, c). The type changing
    between the examples is underlined.}
  \label{fig:simtyping}
\end{widefigure}

As seen, it is possible to mix types with constrained and unconstrained
bit-widths, This is useful as we often know the possible range of external
buses. Determining the ranges of values that derive from those buses are not
always as easy. Therefore, we can let all internal buses and variables of a
program be typed based on observed values while fixing external buses to a
specific size. The type system of SMEIL will then enforce that we do not assign
a larger dynamically determined value to a smaller fixed external value.
% Facilities for
% bit-precise typing also exists in other current SME implementations, but since
% the types aren't built into the language, they become more difficult to enforce

This feature can only be used safely if the following conditions
apply: \begin{inparaenum}[1)] \item All values deriving from input stimuli must
  increase monotonically in relation to the value of the input and \item the
  testing code must ensure that the whole possible range of input stimuli is
  exhausted by test benches.
\end{inparaenum}

To allow easy examination of the types derived from value observations without
having to examine the generated source code, \libsme{} is able to display the
SMEIL program with the modified types in place. That is if the program shown in
\Cref{fig:oktype} was simulated, the resulting code shown in
\Cref{fig:non-violated} or \Cref{fig:violated} would be what is actually shown
to the user.

% A limitation of the current implementation is that SMEIL buses are referenced by
% their bare name, rather than in relation to their position in the instance
% hierarchy. In SMEIL, it is perfectly valid to have multiple buses by the same
% name as long as are not declared in the same scope. However, having several
% {\ttfamily exposed} buses by the same name and referencing them through the
% co-simulation API currently results in undefined behavior. \libsme{} should be
% modified to either disallow multiple {\ttfamily exposed} buses globally, or
% better, to export buses through the API using hierarchical names. The latter is
% already being done by the VHDL code generator.
% \todo{This is not quite correct}

\section{Alternative approaches}
We considered a couple of alternative approaches before settling on this final
design. As written in the introduction to this chapter, co-simulation was
introduced as a method for providing external test inputs to SMEIL programs.
Instead of adding the API for performing co-simulation, we could simply require
that all SMEIL programs expecting external inputs would take these inputs from a
\gls{csv} trace-file generated by simulating an equivalent SME network. In this
scenario, the starting point would be a complete SME network implemented in, for
example, PySME. Simulating this network would generate a trace-file containing a
recording of values sent over buses. Then, the hardware-targeted processes of
the PySME model would be translated to SMEIL and the recorded trace-file would
be used for providing input to the SMIEL model.

%\todo{Why would this be a problem?}
Implementing this model would require less work, but it would only support using
SMEIL as an intermediate language. The reason for this is the following: in
order to generate the required trace file, a complete implementation of the SME
network in a single language is required, thus, it would not be possible to
implement a part of the network separately in SMEIL. Furthermore. the
co-simulation model delegates the responsibility of generating the trace-file to
\libsme{}. Since \libsme{} is also responsible for generating the final VHDL
code, this makes it simpler to ensure that names and order of fields in the CSV
file match those expected by the generated VHDL test bench. If the trace file
was generated externally, it would also prevent changing the \libsme{}
implementation in a way which altered the naming and ordering of fields in the
CSV files. Thus, the co-simulation model is advantageous even when SMEIL is used
as an \gls{il}, and without it, SMEIL would not be usable as a primary
implementation language at all.

\paragraph{External library generation.}
Insetad of implementing an interface to the simulator within \libsme{}, we
considered the alternative approach of generating an external library (for
example in C++) and expose an API similar to the current. However, instead of
loading the \libsme{} library and asking it to load an SMEIL program, we would
load the generated library directly. The advantages of this would be:
\begin{enumerate}
\item We would get two birds with one stone by both providing a method for
  co-simulation and a C++ code generator.
\item The SMEIL code would be compiled to native code and therefore execute
  significantly faster.
\item A library generated from a SMEIL program would be significantly smaller
  than all of \libsme{} and could be distributed independently.
\end{enumerate}
However, the disadvantages would be:
\begin{enumerate}
\item More complex client library integration: A library performing
  co-simulation with SMEIL would first need to compile the SMEIL code to a
  library, then load the library. We could handle this from \libsme{}, but that
  would diminish the third advantage.
\item The implementation of observationally-derived typing (\Cref{sec:typing})
  would become more complicated as the observed types would need to be
  communicated back to \libsme{} in order for them to be used in the code
  generated from the SMEIL program.
\end{enumerate}
    
\paragraph{API considerations.}
The second consideration made was how to actually expose the API. As an
alternative to the current C API, a web-based REST-style API was also
considered. To implement this, \libsme{} would contain a web-server which would
listen to requests from a client library. The advantage of this approach would
be that web-APIs are more ubiquitous that C-style APIs, and thus, they may be
able to support a wider range of clients. On the other hand, we were concerned
with the performance of such an approach as issuing an HTTP request carries a
significantly higher overhead than performing a platform level C-call. Another
concern with the currently chosen approach was whether it was sufficiently
platform-neutral. However, we feel reasonably confident that the current
approach is supported on all major platforms, although only Linux has been
tested.


%%% Local Variables:
%%% mode: latex
%%% TeX-master: "../master"
%%% End:
