\chapter{Installation Instructions}
\label{app:inst}

This appendix contains information on how to install and run the SMEIL system in
order to reproduce the runs shown throughout the thesis. These instructions has
only been tested on a Fedora 27 machine and uses software versions (except
stack) from its standard repositories. We expect them to work on any recent
Linux distribution although we make no guarantees. These instructions are
superseded by any instructions which may appear online



\section{Dependencies}
The required dependencies for building and running are listed below
\begin{itemize}
\item {\ttfamily stack} --- \url{https://haskell-lang.org/get-started}
  {\ttfamily 
\item gcc
\item python3.6
\item perl
\item pip3}
\end{itemize}

\noindent The individual software components referenced throughout this thesis:

\begin{itemize}
\item libsme --- \url{https://github.com/truls/libsme}
\item pysme --- \url{https://github.com/truls/pysme}
\item almique --- \url{https://github.com/truls/almique}
\end{itemize}

\section{Installation}
\subsection{PySME}
\begin{enumerate}
\item Go to PySME directory
  \item Run {\ttfamily pip3 install --user .}
\end{enumerate}

\subsection{libsme}
\begin{itemize}
\item Go to the \libsme{} directory
\item Run {\ttfamily make} (this will take a while)
\item When {\ttfamily make} completes it has created the \texttt{tools/runsme}
  script. Copy this script somewhere in your \texttt{\$PATH}.
\end{itemize}

\subsection{almique}


%%% Local Variables:
%%% mode: latex
%%% TeX-master: "../master"
%%% End:
