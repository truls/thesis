\chapter{Co-Simulation}
\label{sec:cosim}

Without a method for interacting with other languages, SMEIL would not be very
useful. The simplicity of SMEIL can be attributed to its narrow scope: it is
only intended as a hardware modeling language and not for writing
test-benches. For this, the full power of a general-purpose language is needed
as the test-code can be written without hardware-related considerations and
using all available libraries. For example, a test-bench may read an image from
disk or visualize the results of a simulation. Extending SMEIL to be able to
perform such tasks is a substantial undertaking that does not further its
{\itshape raison d'être} as a hardware-modeling
language. Co-simulation~\cite{schloegl2015towards} is the process of two
separate entities (in this case two SME networks) which communicates through
plumbing transparently established by the SME libraries.

\section{The API}
For performing co-simulation with SMEIL, we expose a C API from \libsme{}. The
API is intended to be used by SME implementations for general-purpose languages.

There are three aspects to the API: Firstly it provides a way to enumerate the
buses exported from a SMEIL model. Secondly, it provides calls for reading to-
and writing from bus channels and driving forward the simulation. Finally, it
offers calls for ordering the production of output, such as VHDL code
generation.

A noticeable feature of the API is its support for arbitrary-length
integers. This means that the bit-width of buses used in simulated designs are
not restricted by the register-width of the underlying CPU


When this API is used \libsme{} operates as a ``puppet'', being controlled by
the calling program (the ``puppeteer''). Only buses declared with the
\texttt{exposed} (\Cref{sec:langref}) modifier in SMEIL are accessible through
the \libsme{} API. The calling library drives forward the simulation by calling
a function for ticking the clock and reading writing from/to the exposed buses
of a SMEIL program. In the following section, we give a detailed introduction to
the API.  For brevity, we refer to the implementing SME library (not PySME in
particular) as the ``client'' in the following.

% \begin{figure}
%   \resizebox{.7\linewidth}{!}{
%     \begin{tikzpicture}[>=latex]
%       \coordinate (A) at (2,5);
%       \coordinate (B) at (2,0);
%       \coordinate (C) at (6,5);
%       \coordinate (D) at (6,0);

%       \draw[thick] (A)--(B) (C)--(D);

%       \draw (A) node[above]{libsme};
%       \draw (C) node[above]{Client};

%       \coordinate (E) at ($(A)!.25!(B)$);
%       \coordinate (F) at ($(C)!.1!(D)$);
%       \draw[->] (F) -- (E);
%     \end{tikzpicture}}
% \end{figure}


This approach is conceptually similar to the Verilog Procedural Interface
(VPI)~\cite{dawson1996verilog} which is used for interfacing with Verilog and
VHDL simulators. However, a big advantage of the SMEIL approach is that SME is
used on both sides of the co-simulation. Hence, both the functional and
verification parts of the network act as a single unified entity. Thus, the
programmer does not need to consider integrating different abstract interfaces.


\begin{figure}
  \centerfloat
  \begin{subfigure}[t]{0.40\paperwidth}
\begin{lstlisting}[language=smeil]
proc plusone(in inbus, const val)
    exposed bus plusout {
       val: i32;
    };
{
    plusout.val = inbus.val + val;
}

network plusone_net() {
    exposed bus idout {
        valid: bool;
        val: i32;
    };

    instance plusone_inst of
        plusone(idout, 1);
}
\end{lstlisting}
    
\subcaption{The SMEIL code in \texttt{addone.sme}.}
  \end{subfigure}
  \begin{subfigure}[t]{0.40\paperwidth}
    \begin{lstlisting}[language=python]
from sme import *

class Id(SimulationProcess):
  def setup(self, ins, outs, result):
    self.map_outs(outs, "out")
    self.map_ins(ins, "inp")

  def run(self):
    result[0] = self.out["val"]
    self.out["val"] = self.inp["val"]

@extends("addone.sme", ["-t trace.csv"])
class AddOne(Network):
  def wire(self, result):
    plus_out = ExternalBus("plusout")
    id_out = ExternalBus("idout")
    p = Id("Id", [plus_out],
      [id_out], result)
    self.add(plus_out)
    self.add(id_out)
    self.add(p)

if __name__ == "__main__":
  sme = SME()
  result = [0]
  sme.network = AddOne("", "AddOne",
             result)
  sme.network.clock(100)
  print("Final result was ", result[0])
  \end{lstlisting}
  \subcaption{The corresponding Python code.}
  \end{subfigure}

  \caption{Example code showing interaction between SMEIL (left) and PySME
    (right).}
\label{fig:smeilpy}
\end{figure}

\section{API Reference}
This section documents the public API of the co-simulation interface of \libsme{}
\todo{finish. Actually explain things.}

\subsection{Exported data structures}
\begin{lstlisting}[language=c,multicols=2]
typedef enum Type {
  SME_INT,
  SME_UINT,
  SME_FLOAT,
  SME_DOUBLE,
  SME_BOOL
} Type;

typedef struct SMEInt {
  int len;
  int alloc_size;
  int negative;
  char* num;
} SMEInt;

typedef struct Value {
  Type type;
  union {
    bool boolean;
    SMEInt* integer;
    double f64;
    float f32;
  } value;
} Value;

typedef struct ChannelVals {
  Value* read_ptr;
  Value* write_ptr;
} ChannelVals;

typedef struct ChannelRef {
  char* bus_name;
  char* chan_name;
  Type type;
  Value* read_ptr;
  Value* write_ptr;
} ChannelRef;

typedef struct BusMap {
  int len;
  ChannelRef** chans;
} BusMap;
\end{lstlisting}

\subsection{Public API}
\begin{description}
\item[\texttt{SmeCtx* sme\_init()}]\hfill\\
   Initializes and returns the SME library context.
\item[\texttt{bool sme_open_file(SmeCtx* ctx, const char* file, int argv, char**
    argc);}]\hfill
     Loads an SMEIL file, while applying the supplied arguments to libsme.
\item[\texttt{bool sme_has_failed(SmeCtx* ctx);}]\hfill\\
   Returns true if an operation within the libsme library failed.
 \item[\texttt{char* sme_get_error_buffer(SmeCtx* ctx);}]\hfill\\
   Returns a string containing the error message emitted by libsme. The memory
   pointed to may not be freed except by calling the \texttt{sme_free} function.
 \item[\texttt{void sme_free(SmeCtx* ctx);}]\hfill\\
   Frees the SME library context and related resources.
 \item[\texttt{bool sme_tick(SmeCtx* ctx);}]\hfill\\
   Ticks the clock of an SME simulation synchronously. When this function
   returns, all processes defined within libsme will have run and written to
   their buses
 \item[\texttt{bool sme_finalize(SmeCtx* ctx);}]\hfill\\
   Finalizes a simulation and dumps the recorded trace file (if any) to the file
   system. This function should always be called following the final call to
   \texttt{sme_tick}.
 \item[\texttt{bool sme_propagate(SmeCtx* ctx);}]\hfill\\
   Propagates the values of both internal and external facing buses defined in
   libsme. Run this function before the clock is advanced (by calling
   \texttt{sme_tick}) in the simulation loop and it should be run together with
   any bus propagations that need to be performed by the calling code. When this
   function returns, the values of all buses defined within libsme have been
   propagated.
 \item[\texttt{void sme_integer_resize(SMEInt* num, int len);}]\hfill\\
   When manipulating values of type SMEInt (arbitrary-size integers) the
   \texttt{sme_resize_integer} function will make sure that the memory pointed
   to by Value.value is large enough to hold the number that you intend to
   store. The function takes a pointer to the SMEInt structure and a parameter
   len which is the size of the number to be stored in base 256. This function
   must be called before every direct manipulation of SMEInt.num. For a safer
   interface, see \texttt{sme_integer_store}.
 \item[\texttt{void sme_integer_store(SMEInt* num, int len, const char val[]);}]\hfill\\
   Stores the base-256 representation of an integer in an \texttt{SMEInt}.
 \item[\texttt{void sme_set_sign(SMEInt* num, int sign);}]\hfill\\
   Sets the sign of an \texttt{SMEInt}. Possible values for \texttt{sign} are 0
   meaning the number is positive and 1 for a negative value.
 \item[\texttt{BusMap* sme_get_busmap(SmeCtx* ctx);}]\hfill\\
   Returns a pointer to a BusMap structure containing the exposed buses of the
   SME network. This function is intended to be used by implementers of libsme
   to generate internal representations of their SME buses. It is the caller
   responsibility to free the memory returned by the function by calling
   \texttt{sme_free_busmap}.
\item[\texttt{void sme_free_busmap(BusMap* bm);}]\hfill\\
   Frees a BusMap structure allocated by \texttt{sme_get_busmap}.
\end{description}

\section{Co-simulation using PySME}
In order to show a client implementation of the API, we have extended the PySME
library~\cite{pysme} with support for co-simulation enabling seamless
interaction between SME networks written in Python and SMEIL. In practice,
extending the PySME library was quite straight-forward and required less than a
day of implementation work by a person with expert knowledge of both
code-bases. We expect that a similar effort is required to extend other SME
implementations (such as C\# SME and C++ SME).

As an example, we revisit our trivial \textsc{plusone} network from before, this
time implementing one half of it in Python seen in \Cref{fig:smeilpy}. The
\texttt{@extends} decorator is all that is needed to make the buses exposed from
the SMEIL network available for the Python program. Behind the scenes, \libsme{}
is loaded and the \texttt{addone.sme} file is parsed, typechecked and the
\libsme{} SMEIL simulator is initialized. A SMEIL-defined bus is referenced from
PySME by creating an \texttt{ExternalBus}, providing the name of a bus as its
parameter. The semantics of an {\ttfamily ExternalBus} is identical to those of
a bus defined within Python. Any SMEIL type except strings and arrays may be
passed along a bus. Strings play a very limited role in SMEIL and are therefore
unlikely to be supported. Arrays, on the other hand, are desirable to include in
future extensions. Integers are encoded in base-256 as a sequence of bytes,
allowing arbitrarily-sized integers to be used between co-simulated entities.

When this program is run, the PySME library calls the \libsme{} library for
every cycle to stepwise progress the simulation. During the simulation \libsme{}
may, if asked to do so, record a trace of the communication taking place over
the buses to a file. This trace file is then later used as the data source for
the VHDL test bench which is used to verify the generated VHDL code.

This is a highly flexible model as co-simulation is enabled with minimal
intrusion on existing PySME code. For example, should \libsme{} be extended with
a high-performance simulation backend for SMEIL, existing programs can take
advantage of this without modifications. The implementation of \libsme{} may
even be replaced entirely, as long as the current API is
maintained. Furthermore, it also facilitates an incremental design strategy,
where a Python prototype can gradually be rewritten in SMEIL.

We show more examples of using co-simulation to test SMEIL networks in
\Cref{eval}.

% \section{Runtime representation}
% In order to perform a simulation of  SMEIL network, we transform it to a
% suitable runtime representation. 


\section{Typing Networks Through Simulation}
\label{sec:typing}
\todo{Why do we only have this restricted notion of dynamic typing. Why no
  completely variable types?}  In order to translate SMEIL to a hardware
description, we require that all types in the program are constrained to a
specific bit-width. However, it is often hard to know the optimal bit-width of a
value in advance. In particular, this applies to internal variables whose values
are derived from external inputs. To address this, \libsme{} provides a method
for re-typing a SMEIL program based on values observed during simulation.

% As described in \Cref{sec:typesys}, the only unsized types which exists in SMEIL
% are signed and unsigned integers. This raise the question: Why not have an
% {\ttfamily any}-type which could unify with any other type? A related form of
% the question is: Why distinguish between signed and unsigned unsized integer
% when the signed type is the superset of the two 
\todo{Describe implementation details}
\todo{We promise to mention ranges earlier}
When the simulation is concluded, the observed value ranges are converted to
sufficiently large SMEIL types. The types and observed ranges are then spliced
into the SMEIL AST and the re-typed program is then passed through the type
checker. This ensures that constraints originating from fixed-size types in the
original program are not violated. This process is illustrated in
\Cref{fig:simtyping} which shows how observationally derived types are spliced
into an existing program. \Cref{fig:violated} shows the violation of an existing
constraint in the program. Since the value {\ttfamily c} has the fixed-sized
type {\ttfamily i10}, the program will no longer be valid if {\ttfamily b.chan}
observes values that are 15-bit long. A configuration flag {\ttfamily
  -{}-no-strict-type-bounds} overrides this behavior by considering all types as
unbounded (i.e., {\ttfamily i10} is considered identical to {\ttfamily int}).


\begin{widefigure}
  \centering
  \begin{subfigure}[t]{0.30\linewidth}
    %\centering
\begin{lstlisting}[language=smeil]
proc A ()
 bus b {
  chan: ?{\bfseries\underline{int}}?;
 };
 var c: i10;
{
 c = b.chan;
}
\end{lstlisting}
    \caption{Unconstrained types.}
  \end{subfigure}
  \begin{subfigure}[t]{0.30\linewidth}
    %\centering
    \begin{lstlisting}[language=smeil]
proc A ()
 bus b {
  chan: ?{\bfseries\underline{i5 range 0 to 29}}?;
 };
 var c: i10;
{
 c = b.chan;
}
    \end{lstlisting}
    \caption{Valid.}
  \end{subfigure}
  \begin{subfigure}[t]{0.30\linewidth}
    %\centering
    \begin{lstlisting}[language=smeil]
proc A ()
 bus b {
  chan: ?{\bfseries\underline{i15 range 0 to 30717}}?;
 };
 var c: i10;
{
 c = b.chan;
}
\end{lstlisting}
    \caption{Invalid.}
    \label{fig:violated}
  \end{subfigure}
  \caption{Shows a process entering the simulator with an unconstrained type (a)
    and examples of two possible resulting programs (b, c). The type changing
    between the examples is underlined.}
  \label{fig:simtyping}
\end{widefigure}

As seen, it is possible to mix types with bounded and unbounded bit-widths, This
is useful as we often know the range of external buses. Determining the
ranges of values that derives from those buses are not always as
easy. Therefore, we can let all internal buses and variables of a program be
typed based on observed values while fixing external buses to a specific
size. The type system of SMEIL will then enforce that we do not assign a larger
dynamically determined value to a smaller fixed external value.
% Facilities for
% bit-precise typing also exists in other current SME implementations, but since
% the types aren't built into the language, they become more difficult to enforce

This feature can only be used safely if the following conditions
apply: \begin{inparaenum}[1)] \item All values deriving from input stimuli must
  increase monotonically with the value of the input and \item the testing code
  must ensure that the whole possible range of input stimuli is exhausted by
  test benches.
\end{inparaenum}

To allow easy visualization of the types derived from value observations without
having to examine the generated source code, \libsme{} is able to display the
SMEIL program with the modified types in place.

A limitation of the current implementation is that SMEIL buses are referenced by
their bare name, rather than in relation to their position in the instance
hierarchy. In SMEIL, it is perfectly valid to have multiple buses by the same
name as long as are not declared in the same scope. However, having several
{\ttfamily exposed} buses by the same name and referencing them through the
co-simulation API currently results in undefined behavior. \libsme{} should be
modified to either disallow multiple {\ttfamily exposed} buses globally, or
better, to export buses through the API using hierarchical names. The latter is
already being done by the VHDL code generator.
\todo{This is not quite correct}

\section{Alternative Approaches}
We considered a couple of alternative approaches before settling on this final
design. As written in the introduction to this chapter, co-simulation was
introduced as a method for providing external test inputs to SMEIL programs.
Instead of adding the API for performing co-simulation, we could simply require
that all SMEIL programs expecting external inputs would take these inputs from a
CSV file from the simulation of an equivalent SME network. In this scenario, the
starting point would be a complete SME network implemented in, for example,
PySME. By simulating this model, a trace file containing a recording of values
sent over buses would be generated. Then, the hardware-targeted processes of the
PySME model would be translated to SMEIL and the recorded tracefile could be
used for providing input to the SMIEL model.

\todo{Why would this be a problem?}. Implementing this model would require less
work, but it would only support the usage of SMEIL as an intermediate language
and not as a primary implementation language: in order to generate the required
trace file, a complete implementation of the network in a single language is
required. Furthermore. the co-simulation model delegates the responsibility of
generating the trace-file to \libsme{}. Since \libsme{} is also responsible for
generating the final VHDL code, this makes it simpler to ensure that names and
ordering of fields in the CSV file matches those expected by the generated VHDL
test bench. If the trace file was generated externally, it would also prevent
changing the \libsme{} implementation in a way which altered the naming and/or
ordering in the CSV files. Thus, tje co-simulation model is advantageous even
when SMEIL is used as an \gls{il}.

\paragraph{External library generation}
As an alternative to implementing an interface to the simulator within
\libsme{}, we considered the alternative approach of generating an external
library (for example in C++) and expose an API similar to the current. However,
instead of loading the \libsme{} library and asking it to load an SMEIL program,
we would load the generated library directly. The advantages of this would be:
\begin{enumerate}
\item We would get two files in one swat by both providing a method for
  co-simulation and a C++ code generator.
\item The SMEIL code would execute significantly faster.
\item A library generated from a SMEIL program would be significantly smaller
  than all of \libsme{} and could be distributed independently.
\end{enumerate}
However, the disadvantages would be:
\begin{enumerate}
\item More complex client library integration: A library performing
  co-simulation with SMEIL would first need to compile the SMEIL code to a
  library, then load the library (we could handle this from \libsme{}, but
  that would diminish the third advantage)
\item The implementation of observationally-derived typing (\Cref{sec:typing})
  would become more complicated as the new type
\end{enumerate}
    
\paragraph{API considerations}
The second consideration made was how to actually expose the API. As an
alternative to the current C-style API, a web-based REST-style API was also
considered. For this, \libsme{} would contain a web-server which would
listen to requests from a client library and act accordingly. The advantage of
this approach would be that web-APIs are more ubiquitous that C-style APIs, and
thus, they may be able to support a wider range of clients. On the other hand,
we were concerned with the performance of such an approach as issuing an HTTP
request carries a significantly higher overhead than performing a platform level
C-call. Another concern with the currently chosen approach was whether it was
sufficiently platform neutral. However, we feel reasonably confident that the
current approach is supported on all major platforms although Linux has only
been tested.


%%% Local Variables:
%%% mode: latex
%%% TeX-master: "../master"
%%% End:
