\section{Motivations for custom hardware}
% \todo{Write this}. Give a brief overview of a hardware design workflow and give
% an overview of especially VHDL as this is discussed here repeatedly.

% %Basic hardware architecture -- gate level

% %FPGAS -- Reconfigurability, LUTs, BRAM, m.m.

% %Programming models -- HDLs (VHDL), ..

% \todo{This needs a shitload of references}

% This section is an attempt to give a more in-depth motivation of why 

%\subsection{A brief history}
Digital circuits are fundamentally a collection of logic gates which are
connected in a specific configuration to fulfill a particular purpose. In
\glspl{ic}, this configuration is hard-wired -- etched into
silicon using a lithographic process. An example of a hard-wired \gls{ic} is the
common \gls{cpu}. \glspl{cpu} are highly versatile devices, capable
of computing anything computable. However, in addition to their low price, this
versatility is their main advantage and they excel at nothing else.
% , it also mean that they excel at nothing.
% \todo{mention von
%   neumann}\footnote{To keep this short, we don't discuss the widely available
%   Single Instruction Multiple Data (SIMD) instruction sets such as 3Dnow!, MMX,
%   SSE and AVX. These implement data-level parallelism in a restricted
%   manner}.
Since the advent of the first microprocessor, almost 5 decades ago, this problem
has been widely recognized. Therefore, a steadily increasing amount of
special-purpose hardware is being added in order to relieve the main CPU
pipeline of common and computationally intensive tasks. An early, and highly
successful, example of this is the introduction of co-processors for performing
floating-point calculations. As personal computers were increasingly being used
for tasks relying heavily on floating-point arithmetic, emulating this in
software increasingly became a limiting factor for performance. To solve this,
specialized hardware units, known as Floating Point Units (FPU) were added to
significantly speed up floating point calculations compared to what was possible
using software emulation. Numerous similar examples exist, for example, the
AES-NI instruction set built into recent CPUs, providing access to specialized
circuitry for performing encryption and decryption using the ubiquitous Advanced
Encryption Standard (AES) algorithm.

The current use of specialized hardware only scratches the surface of
applications for which it would be beneficial. For example, by offering a
significantly improved performance-per-watt
ratio~\cite{fowers2012performance}. This is especially important in the light of
the ever-increasing power consumption of data centers~\cite{avgerinou2017trends}
which counter global efforts to decrease emissions. In an ideal world, everyone
would have cheap and easy access to creating hardware specialized for their
particular application. Unfortunately, such hardware, known as \glspl{asic}, is
extremely expensive and time-consuming to design and put into
production. Furthermore, a very large number of units has to be ordered before
achieving a reasonable cost-per-unit. Finally, mistakes are expensive since once
the hardware is made it can never be changed.

% Another well known example of special-purpose hardware are Graphics Processing
% Units (GPU) introduced to, as the name suggests, accelerate graphics in
% computers. Therefore, GPUs are optimized for workloads that are ubiquitous in
% graphics processing such as bulk operations on arrays. Therefore, GPUs are
% massively parallel machines, capable of performing a single operation, in
% parallel on a large set of data.


% In 1965 Gordon Moore famously stated that the number of transistors on a single
% IC would double every year. This is commonly known as Moore's Law and has ensured
% a steady increase in performance ever since. Today, Moore's law i

\glsreset{fpga}
\subsection{\Glspl{fpga}}
Reprogrammable computing, of which Field Programmable Gate Arrays (FPGAs) are
the only prominent example, offers an attractive compromise between \glspl{asic} and
more general devices such as \glspl{cpu} or \glspl{gpgpu}. While not nearly as good as
\glspl{asic}~\cite{kuon2007measuring}, they can provide significant improvements in
both performance and power consumption at a much lower cost.

\glspl{fpga} are \glspl{ic} which allow their circuits to be changed after
manufacture (hence, they are programmable in the field) and can, therefore, be
reconfigured to fulfill any purpose. Of course, the circuitry on the chip cannot
be physically changed, so \glspl{fpga} consists of an array of logic blocks
which has configurable interconnects. The reprogramming of an \gls{fpga} happens
by reconfiguring switches which determines the signal path. Since the individual
components on an \gls{fpga} are fine-grained, they can be reconfigured for any
computational purpose by rewiring signal paths.

\subsection{The VHSIC Hardware Description Language (VHDL)}
As briefly mentioned in the introduction, \glspl{fpga} are usually programmed
using Hardware Description Languages (HDLs) such as \gls{vhdl} or Verilog. We
will focus on VHDL here, since that is the current target of SMEIL code
generation.

VHDL was originally developed in the 1980's as a formal specification language
for documenting hardware designs, with the first IEEE standard being released in
1987~\cite{ashenden2001vhdl}. Initially, it was purely a specification language,
not intended to be run by computers. It was only later that simulators for VHDL
were developed allowing designs written in the language to be automatically
tested. Even later, tools were developed for performing {\it syntehsis} of VHDL
which meant that a VHDL hardware description could be transformed to an actual
circuit which could be implemented on hardware.

The standard of VHDL which is most widely supported by tools was released in
1993. This is despite the fact that an updated version of the standard was
released in 2008. However, even the most recent iteration of the standard is
fundamentally unchanged from the initial version. Therefore, VHDL has not
benefited from decades of research in programming language productivity. As also
written in the introduction, this becomes painfully obvious when VHDL is used
for anything other than writing the actual hardware description. Most hardware
design workflows today use traditional HDLs both for writing the actual hardware
implementation and the corresponding test-bench. Thus, the inadequacy of
traditional HDLs for fulfilling the latter purpose creates both a productivity
issue and a high barrier of entry for potential hardware designers.

Poor hardware design languages are not the only problem causing a high barrier
of entry into the world of custom hardware. However, it is the problem that we,
as software developers, are best equipped to help solving.
% \todo{Grammarly wants
%   ``help to sovle''.  Maybe rewrite sentence?}.

% Poor languages for specifying hardware is not the only issue establishing a high
% barrier of entry into hardware design, however, it is the one that we, as
% software developers, are able to help reducing.

%\todo{finish}
%VHDL was originally developed as 

%%% Local Variables:
%%% mode: latex
%%% TeX-master: "../master"
%%% End:
