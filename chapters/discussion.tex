\chapter{Discussion}

In the previous chapters, we have described a new language for representing SME
networks, SMEIL and how its implementation fits into the greater whole
\todo{rewrite}. The questions that are left to answer now are, what have we
achieved by doing so? Is SMEIL suitable for the current SME ecosystem, and
finally,

\section{Completeness of SMEIL in relation to SME}
In this section, we try to provide some insight into whether the SMEIL provides
a {\itshape complete} representation of the SME model. Thus, a complete SME
implementation is \begin{inparaenum}]a)] \item able to construct networks from 
\end{inparaenum}

A complete SME implementation implements all the key elements of the SME model
including
\begin{itemize}
\item Independent processes
\item Buses
\item Synchronous communications
\item Ability to create compositions of networks.
\end{itemize}
as

As we have shown in several examples. 

\section{Usability of SMEIL}
As a language for implementing simple SME networks, SMEIL is already functional
and usable. We have seen this ... . An issue which we have not discussed is how
well SMEIL scales for larger networks since we have only looked at small
examples. For this, we argue that the scalability of SMEIL is similar to that of
C\# SME which has previously been used to create large-scale SME
implementations. This is because the primary structural components of the C\#
SME implementations maps to SMEIL in a straight-forward manner.

A shortcoming of SMEIL in its current state is that we do not yet support all of
the features of the C\# SME implementation, however, doing so was not within the
scope of the thesis in the first place.

We believe that

Co-simulation is a frequently used technique for verifying hardware designs,
however, 

Our approach for determining types based on observed input is, to the best of
our knowledge, unique within the landscape of Hardware Description
Languages. Several 

Several other approaches

Another aspect which is outside the scope of the thesis is actually implementing
the generated VHDL code on hardware. However, it is still an important aspect to
discuss since the primary purpose of writing SME models is to, as previously
mentioned, eventually create a hardware description. 

% \section{Relation with other HDLs}
% Discuss how SME and specifically SMEIL relates to other similar ``simple''
% hardware description languages. \todo{Should this be here?}

% \section{Related co-simulation approaches}
% \todo{Should this be here}

\section{Comparison with ``state-of-the-art'' SME}
Write about: Internal buses: We have language support based on inference from
usage, but not implemented. Arrays as busses



\section{Target Language Support}
We have currently only implemented a single language backend for SMEIL:
VHDL. However, compiling SMEIL to other languages, such as C++ (see
\cite{skovhede2017c++}) is also desirable. Furthermore, \libsme is written with
support for multiple target languages in mind, and the infrastructure to add
another target language is in place, but unfortunately time constraints did not
allow us to do it. Specifically in relation to other HDLs such as Verilog. Most
new high-level HDLs is able to generate at lest VHDL and Verilog. SME is
currently only translatable to VHDL, however, adding support for Verilog is
entirely possible. To support this claim, we point
to \begin{inparaenum}[a)]\item The large number of tools (e.g. \todo{cite})
  )already supporting both languages and \item numerous \end{inparaenum} \todo{cite} \todo{Why do
    we think its simple to add another language?} We have not assessed the
  practicality of transforming SMEIL to languages following other paradigms,
  such as functional languages or alternative hardware description
  languages. However, such languages is outside the scope of the thesis.

%%% Local Variables:
%%% mode: latex
%%% TeX-command-extra-options: "-enable-write18"
%%% TeX-master: "../master"
%%% End:
