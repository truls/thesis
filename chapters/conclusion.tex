\chapter{Conclusions}

% \section{Related work}

% \section{Future Work}

% A key assumption of this work is that we assume that input values explore the
% value ranges of SME buses. In many cases, this assumption may be impractical as
% the quality of test-cases has an enormous impact on the correctness and
% usefulness of the resulting implementation. To address this, methods for
% inferring the value ranges of SME buses using purely static means may also be
% useful.

% Allow \gls{vhdl} code to be embedded within \gls{smeil} and simulate by linking
% the simulation to a \gls{vhdl} simulator using \gls{vhpi}. This is similar to
% work done by cocotb and clash master

\section{Related Work}
\label{sec:retwork}
% Work related to SMEIL centers around two different areas; high level languages
% for implementing hardware designs mention.

% Xilinx HLS attempts to infer concurrency from sequential C programs guided by
% annotations and use that to generate hardware descriptions. A drawback of this
% approach is that the quality of the resulting hardware implementation depends
% greatly on the quality of the added annotations. SME takes a different approach
% in being explicit about the concurrency of the resulting hardware design.

In addition to the HLS approaches mentioned in the introduction, several
alternative hardware design modeling tools have been proposed both in the
industry and in academia. Furthermore, a number of approaches to replace test
benches written in traditional HDLs has been proposed.

MyHDL~\cite{myhdl} is a Python-based HDL, essentially a DSL embedded in
Python. It is intentionally implemented as a high-level version of traditional
HDLs while enabling Python to be used for test-benches. Since it inherits its
worldview from traditional HDLs, it has a different goal than SME which provides
an abstraction through the SME model.

Cx~\cite{cxlang} is a dedicated DSL for writing hardware designs. The Cx
language has several similarities with SMEIL, for example, the type system. Like
SME, it allows the programmer to explicitly control concurrency by building
networks of processes. However, despite claims on its website, Cx is a
proprietary language requiring a license for long-term use giving it a high
barrier-of-entry.

C\textlambda{}aSH~\cite{wester2015transformation} and Lava\cite{bjesse1998lava}
are two Haskell based approaches with different philosophies: Lava is a Haskell
design pattern (with several implementations e.g., \cite{gill2009introducing})
for specifying composable circuits at the gate-level. The extremely low-level
approach means that it is targeted towards replacing and formalizing certain
low-level uses of HDLs rather than as a general high-level hardware modeling
tool. C\textlambda{}aSH, on the other hand, transforms a subset of high-level
Haskell code to HDLs. This requires concurrency inference, but this is simpler
to do for a purely functional language, such as Haskell, compared to an impure
imperative language, such as C.

A more recent approach~\cite{aronsson2017hardware} also uses Haskell, but only
as a host for an Embedded DSL. This EDSL translates to both VHDL and C, enabling
the programmer to trivially change which parts of her program that runs on the
CPU or the FPGA. The library automates setting up AXI interconnects between the
CPU-part and the FPGA-part of the code. The advantage of this approach is that
it enables simple hardware-software co-design. The primary disadvantage of this
approach, is that the CPU code must also be written in the DSL. For many
applications, this can be overly restrictive since reuse of existing code and
common libraries is not possible.

CAPF~\cite{serot2011implementing}, Pyrope~\cite{pyrope} and
Chisel~\cite{bachrach2012chisel} are HDLs which provide data-flow based design
models. CAPF and Pyrope are independent languages while Chisel is an EDSL in
Scala. The data layouts that are good fits for these languages are also
expressible using SME, albeit less elegantly. However, problems which are best
represented as a sequential algorithm can be a poor fit for the data-flow
paradigm.

The Coroutine Co-Simulation Test Bench (CoCoTB)~\cite{cocotb} also implements a
notion co-simulation between Hardware Descriptions and Python (A General-Purpose
language). Using the Verilog Procedural Interface (VPI) which (despite the name)
is implemented by both VHDL and Verilog simulators. This library presents a
significant advantage over writing test benches exclusively in HDLs, however,
the relative complexity of the VPI interface leaks into the CoCoTB interface,
requiring a non-trivial amount of boilerplate code. Furthermore, it does not
directly address the productivity issues associated with traditional HDLs and
does not offer the unified simulation model used in SME.


\section{Future Work}
Other SME implementations, C\# SME in particular (see
e.g.~\cite{skovhede2018statemachine}) have evolved in parallel with the
development of SMEIL. Therefore, these are more comprehensive and support a
wider range of features. Since SMEIL is, as previously mentioned, intended to
serve as the only target language for SME, SMEIL should obviously be brought
on-par with other existing SME implementations. In the present work, a
substantial amount of compiler-infrastructure groundwork has been laid, making
these improvements a natural continuation of future SMEIL developments.

As the primary target for SMEIL is hardware, VHDL is the only code generation
backend currently implemented. However, we intend to provide a wide range of
language backends~\cite{skovhede2017c++} is another part of catching up with
other SME implementations. For example, generating C++ code can make it possible
to use SME programs with other software as a library and provide significantly
faster simulation than the current interpretation-based approaches are able to
offer.

All SME implementations currently target a single clock domain. Future efforts
should be made towards supporting multiple clocks, running at different speeds.

In some cases, SMEIL offers an insufficient amount of control over the generated
VHDL code or the generated code may simply be inefficient compared to
hand-optimized VHDL code. For this, we should allow inlining VHDL inside SMEIL
by adding language constructs to specify how SME buses should be connected to
VHDL signals. The simulation of such mixed code can be performed by running VHDL
parts in a VHDL simulator.

Hardware-software co-design is an area that is actively researched. The idea is
that specialized hardware is designed in parallel with corresponding software
such that each part of the design can be handled in either hardware or software
depending on what is best suited. Such heterogeneous designs require code for
setting up the communication between the hardware and software parts. SME is
well suited for this and we should, therefore, try to develop a mechanism
allowing communication between SME networks running on different devices where
the actual generation of the communication interfaces is handled transparently.

The presented approach for automatically typing SMEIL networks based on observed
input makes the assumption that the complete possible space of input values is
explored by the testing stimuli. The downside of this approach is that this
assumption may be hard to fulfill. To address this, the current approach could
be augmented by integer range analysis for proving the observed ranges.

To improve the user-friendliness and capabilities of SMEIL, there is a wide
range of language features that we would like to add. A non-exhaustive list
follows:
\begin{itemize}
  \item In practice, not being able to add declarations, such as constants,
    enumerations and functions, at the top-level of an SMEIL program proved too
    restrictive. This should be added.
  \item A syntax for direct bit manipulations. Currently, bit manipulation can
    only be performed through bitwise operators in a similar manner to C. It
    would be convenient to have an array-like syntax for achieving the same
    thing, for example similar to Python's array slicing feature.
\end{itemize}

\section*{Conclusion}
We have presented SMEIL, a DSL for implementing SME networks and demonstrated
its practical use through several examples. Although we have focused on using it
as a primary implementation language for SME networks, it is also usable as an
intermediate language for other SME implementations. SMEIL is based on the
structural components of the SME model and provides a high-level C-like syntax
with constructs commonly found in general-purpose imperative languages. This is
needed in order to ensure that SME networks implemented in general-purpose
languages can be translated without requiring sophisticated transformations.

The type-system presented supports bit-precise types which is important for a
hardware-targeted language. However, the requirement to specify a fixed
bit-width for all types is sometimes impractical. Instead, arbitrary-length
types may be specified which are then constrained based on values observed
during simulation.

Simulation of SMEIL is performed in a manner which provides a cycle-accurate
representation of the resulting hardware. During the simulation, a trace of
channel communications is recorded. SMEIL compiles to readable VHDL code which
can be used for a subsequent hardware implementation. Additionally, a test-bench
is generated which can be used to verify the correctness of the generated
code. The test-bench uses the trace recorded during simulation to allow
continuous verification of the generated code even following manual refinement.

For testing SMEIL networks directly, an interface is provided for performing
co-simulation with SME networks written in general-purpose languages. This
approach proved highly successful in practice.

The presented language and its implementation does not yet provide the full
feature-set of other, more mature, SME implementations. In spite of this, we are
optimistic about its future prospects, regardless of if these include use as an
independent DSL or as a primary implementation language for SME.

%%% Local Variables:
%%% mode: latex
%%% TeX-master: ../master
%%% TeX-command-extra-options: "-enable-write18"
%%% End:
