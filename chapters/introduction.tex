\chapter{Introduction}

Special-purpose hardware has a wide range of different uses and can provide a
significantly improved performance-to-watt ratio compared to \glspl{gpgpu} and
\glspl{cpu}. However, the prevalence of such hardware is limited, in part, by
poor design tools. Traditional hardware design workflows utilize
%{\itshape Hardware Description Languages} (HDLs)
\glspl{hdl} such as \gls{vhdl} or Verilog
which require the programmer to specify the hardware design at a very low
level. While this enables complete control over the resulting hardware, the
productivity sacrifice is significant when compared to using general-purpose
languages for writing software. Additionally, all aspects of a hardware design
are often written in a HDL, including code for testing and
verification. Traditional \glspl{hdl} are fundamentally unfit for performing tasks
commonly needed for simulating input for a design, such as reading and decoding
an image file.% Performing them are tedious at best and impossible at worst.

In the past few decades, there has been a significant interest in tools that
improve the productivity of hardware design workflows. Vendors of reconfigurable
hardware have focused primarily on \gls{hls}. These
utilities transform algorithmic descriptions written in a general-purpose
language to an HDL-description that can be implemented on hardware. The source
languages for the most common \gls{hls} tools are C (e.g. Vivado HLS~\cite{vivadohls})
and OpenCL (e.g. Altera OpenCL~\cite{aocl}).

Hardware is inherently parallel, and utilizing this parallelism is imperative
for achieving good performance. Therefore, efficiently transforming sequential C
code to a hardware description requires inferring parallelism in a similar
manner to, for example, OpenMP. To control the transformation, the programmer is
required to add annotations to the C program. The quality and performance of the
resulting hardware implementation depend greatly on the aptitude of the
programmer to add these annotations correctly. This requires a deep
understanding of the transformation process and the underlying architecture of
the targeted hardware. The difficulties of creating auto-parallelizing compilers
for impure general-purpose languages are and arise well-known in particular due
to the challenges of resolving data
dependencies~\cite{banerjee1976data,bodik2000abcd}. HLS utilities provide no
revolutionary improvements in this regard and thus have a tendency to retain
major sequential parts of the original program resulting in an inefficient
hardware design.

Transforming OpenCL programs to hardware descriptions is a related scheme which
is currently gaining popularity. This option seems more attractive as OpenCL is
already an explicitly parallel language targeting heterogeneous computing
platforms. However, OpenCL code needs to be tuned specifically to each target
platform in order to achieve optimal performance~\cite{chen2012using}. Most
existing OpenCL programs are written with \glspl{gpgpu} in mind. Thus, these programs
must be rewritten to perform optimally on \glspl{fpga}, again requiring heavy use of
annotations. This reduces the portability and productivity advantages of OpenCL.
% Furthermore, OpenCL is most efficient when
% applied to problems which exhibit a high-degree of data parallelism, but
% problems exhibiting other kinds of concurrency are more difficult to
% model.\todo{Is this right?}.
Furthermore, the OpenCL computing model requires the presence of a host device
which makes it unsuitable for creating completely independent hardware
components.

To approach this problem from a different angle, the \gls{sme}
model\cite{vinter2015bus,vinter2014synchronous} has been introduced. SME is
similar to \gls{csp}~\cite{csp}, but replaces the rendezvous-style communication
of \gls{csp} with globally synchronous message passing between processes driven
by a hidden clock\footnote{An extended introduction to the SME model is given in
  \Cref{sec:sme}}.
% This communication model mimics hardware where all signal propagations
% are also globally synchronous, driven by a clock.
This allows the programmer to be explicit about concurrency, using a model which
closely resembles signal propagation in hardware. Thus, SME simplifies
performance reasoning compared to the HLS approaches described above which
relies on inferred concurrency.

As the implementations of SME has advanced, SME has been utilized to create
several successful hardware designs which have been implemented on FPGAs. For
example, a MIPS processor implemented in SME was successfully synthesized and
implemented on an FPGA~\cite{johnsen2017thesis}. These achievements have
motivated and encouraged the continuing development of the model and related
utilities, although we do not claim that it has reached the level of maturity of
the HLS approaches previously mentioned.

SME by itself is just a model, which is not tied to a specific programming
language or implementation. Currently, libraries for implementing SME models
exist for the general-purpose languages C++~\cite{asheim2015},
C\#~\cite{skovhede2016building} and Python~\cite{asheim2016vhdl}. The latter two
have code-generation backends targeting \gls{vhdl}. Maintaining feature-parity
between these independent implementations proved infeasible due to the
code-duplication involved. This created a demand to unify the common backend
components of the divergent code bases of the two SME implementations. To
achieve this, a common intermediate language for SME networks was
needed. Additionally, combining SME networks written in different source
languages was also a desired feature. While we could feasibly introduce an
interface allowing this between Python and C\#, the number of required
interfaces increase exponentially for every language added. Having a common
intermediate language would make this integration simple.

This thesis introduces the SME Implementation Language (SMEIL) and its
accompanying implementation, \libsme{}~\cite{libsme} (sometimes referred to as
``the compiler'').
% SMEIL is an attempt to provide the best of both worlds
% by separating the actual hardware description and assigning them to the best
% suited languages.
SMEIL is a specialized language, featuring a familiar C-like syntax and
structural constructs which are deeply rooted in the SME model. Furthermore, it
provides a type-system which is tailored for hardware-specific subtleties that
are difficult to express in general-purpose languages without deviating from
established paradigms. An explicit design goal of SMEIL is to provide a simple
and straight-forward mapping of code structures commonly found in imperative
general-purpose languages. For testing designs implemented in SMEIL,
general-purpose languages are well suited since their full range of available
libraries can be utilized. Therefore, \libsme{} provides a simple and
language-independent API allowing SME implementations written for
general-purpose languages to communicate with SME networks written in SMEIL.

Although SMEIL was initially intended as an intermediate language target for
existing SME implementations, the resulting language
%has additionally proven to
%be
is also usable as an independent primary implementation language for SME
models. The remainder of the thesis will primarily describe the language from
this perspective. To show its use as an intermediate language, we have adapted
our previous implementation of a Python SME to VHDL compiler~\cite{almique} to
output SMEIL instead of VHDL directly. This is discussed in \Cref{sec:smeilil}.


\section{Motivations for a SME DSL}
\label{sec:smemot}

Initially, we considered just creating a common Abstract Syntax Tree (AST)
representation. This approach would focus on generalizing the existing ASTs
already used internally by the PySME and C\# SME to VHDL transformers. An
advantage of this strategy is that it carries a smaller implementational burden
compared to creating a dedicated language. However, no simple and established
frameworks exist for formally specifying an AST in a language-neutral way. A
representation without a corresponding concrete syntax would also be difficult
to understand and reason about, making it hard to verify the correctness of the
generated intermediate code.

%At this point,
These observations meant that the resulting language got a concrete syntax
%of its own
and fulfilled the original design goal of providing a direct mapping of
constructs from common general-purpose languages. The reason for this design
goal was to ensure that adding new SME frontends would be as simple as possible,
relieving them of performing sophisticated transformations. Thus, SMEIL
inevitably became an independent DSL suitable as a primary implementation
language for SME models. Exploring the concept of an independent SME DSL is
interesting for a number of reasons. In particular, A DSL allows concise and
elegant expression of concepts present in the target domain---the
domain-specific needs of hardware are not considered in the design of
general-purpose software languages.

\paragraph{C\# SME.}
Translating SME models written in C\# is comparatively straight-forward since
language properties can be statically specified and are enforced by the
compiler. For example, if we declare a variable as being constant, we can be
sure that its value will never change beyond its initial assignment and
fixed-length arrays may be explicitly created as such. Likewise, when we declare
a variable to be of a certain type, we can be sure that it will keep that type
throughout the program. Being able to specify such restrictions is immensely
useful as the transformation target (a hardware description), is also
static. However, this also means that hardware-targeted SME models written in
C\# contains a lot of declarative ``noise'' required to confine the C\# language
to a feature set which is possible to implement on hardware. Since the C\#
syntax cannot be extended to natively declare SME elements, annotations are
required to inform the SME runtime and translation system about how a C\# object
should be interpreted.


\paragraph{PySME.}
The situation is different for languages such as Python. The key selling point
of Python is that it is a high-productivity language which is simple to use. It
largely owes these attributes to the fact that it is a dynamic
language. However, this makes it challenging to determine the static properties
needed in a hardware description from a Python program without imposing a heavy
annotational burden. Furthermore, building upon our C\# example from before,
ensuring that a Python variable retains its type throughout program execution
require either sophisticated analyses or strong programmer discipline. Being
able to provide such guarantees is a prerequisite for performing a semantically
unchanged transformation. We base these assertions on our previous experience
building a Python SME to VHDL compiler~\cite{asheim2016vhdl}. While we were able
to transform SME networks written in Python to VHDL, the programmer could only
use a narrow and strictly specified subset of Python in the hardware-targeted
processes. The addition of unfamiliar features (annotations), and required
re-learning of semantic assumptions, reduces the advantage of Python from the
perspective of an experienced Python programmer. It is certainly possible to
improve on our previous attempt at transforming Python. However, this is a
significant effort which does not directly contribute to the capabilities of SME
as a hardware design utility.

% An extreme example of why Python is not always a good ideas is SysPy which uses
% python in a way that almost word-for-word mirrors VHDL and its idiosyncrasies;
% this is exactly what SME is trying to avoid!

% To summarize these assessments, our underlying conjecture is that the underlying 
\vspace{1em} The key advantage of writing SME models in a general-purpose
language is that test code can utilize the full range of libraries available for
that language. It is crucial that this advantage is preserved for SME networks
written in SMEIL. We explain how this is achieved in \Cref{sec:cosim}. A common
objection against DSLs is the requirement of learning a new and unfamiliar
language. However, the SME model itself needs to be learned in any case and the
additional overhead SMEIL imposes, is minimal. From first-hand experience, a
student of computer science familiar with CSP, but not SME, was able to start
writing simple SMEIL programs after just a few hours of introduction.

Due to its origins as an intermediate language, its syntax is not the
friendliest. The syntax attempted to strike a balance between being simple to
parse while not being completely unreadable for humans.

\section{Limitations}
This thesis does not discuss the low-level details of hardware design beyond a
brief introduction to hardware design workflows. As previously mentioned, the
feasibility of SME as a hardware design tool has already been established
through previous successful implementations. All of our references to hardware
design are based on these previous experiences. The problems addressed in this
thesis are purely related to the SME model and the results of our work does not
alter the {\itshape fundamental} qualities of SME as a hardware design tool.


\section{Contributions}
We summarize the contributions of the thesis as follows:

\begin{itemize}
\item We present a new language for implementing SME networks. The language is
  suitable both as a primary implementation language for SME networks and as an
  intermediate language for other SME implementations.
\item We provide a way to test models written in the language using a
  co-simulation approach.
\item We provide a method for deriving the minimally required bit-widths of
  wires in the final design from the observed range of values assigned to them
  during simulation.
\item We demonstrate an implementation of the above points in addition to VHDL
  code generation from designs written in the introduced language.
\end{itemize}

\noindent A paper based on this thesis has been submitted for publication as

\begin{center}
\begin{minipage}{0.8\textwidth}
  T. Asheim, “SMEIL: A Domain Specific Language for Synchronous Message Exchange
  Networks”. In: {\itshape Proceedings of Communicating Process Architectures
    2018} (2018)
\end{minipage}
\end{center}

\section{Notation and Definitions}
We frequently refer to hardware-design nomenclature: A {\itshape test bench} is
a piece of software used for testing a hardware model by providing input data
and verifying its output. {\itshape Synthesis} is the process of transforming a
hardware-model written in a HDL to an actual description which can be
implemented on hardware. We will occasionally refer to ``assigning a
value''. The ``value'' here may, unless specified otherwise, be any assignable
SMEIL construct (either a variable or a bus channel).

% \section{Thesis Structure}
% The remainder of the thesis is structured as follows.
% \todo{finish}


% \section{Contributions}
% I summarize the contributions of this thesis as follows.

% \begin{itemize}
% \itshapeem I present a new Domain Specific Language, \gls{smeil}, which supports
%   implementing \gls{sme} networks. This language is suitable both as a primary
%   implementation language for \gls{sme} networks and as an intermediate language
%   which may be targeted by other implementations of \gls{sme}. This language can
%   be used as part of several design workflows. It may be used both in complete
%   isolation where the \gls{smeil} program alone produces a \gls{vhdl} hardware
%   description. It can also be used in conjunction with other languages.

% \itshapeem This language is presented in the context of the history of \gls{sme} and
%   we present its tradeoffs related to lessons learned and choices made by
%   previous \gls{sme} implementations.

% \itshapeem We show how networks written in \gls{smeil} can interface with \gls{sme}
%   frameworks written in general-purpose language, providing what we refer to as
%   co-simulation. By doing this, we mitigate the poor separation-of-concern
%   (described above) applying to both dedicated \gls{hdl}s and
%   \gls{gpl}'s. Through examples, we underpin the generality of this interface.

% \itshapeem We lay the building blocks for creating libraries of reusable \gls{sme}
%   components by presenting a module system for \gls{smeil}. In particular, we
%   establish the generality of \gls{smeil} by showing that such modules may be
%   combined regardless of the source language that produced the \gls{smeil}.

% \itshapeem We substantiate the feasibility of \gls{smeil} as an intermediate language
%   by showing that the high-level nature of \gls{smeil} easily supports
%   translation from and into several different languages.

% \itshapeem We argue for the advantages of using a \gls{dsl} over general-purpose languages
% by...

% \itshapeem As evidence, we supply a core compiler and simulator implementation,
%   written in Haskell, together with related library extensions to the PySME
%   library and Almique which embodies the qualities listed above. Furthermore,
%   C\# \gls{sme} was kindly extended by Kenneth Skovhede to support translation
%   to \gls{smeil}. The implementation is complete and functional and strives to
%   be of production quality. We spare the reader from further
%   implementation-specific details as these are often the least interesting parts
%   of a report like this. However, we refer to the supplied documentation for
%   details about how to use the supplied code.

% \itshapeem We show the suitability of the language for its intended purpose by
%   showing the simplicity of which a variety of different preexisting \gls{sme}
%   applications may be implemented in \gls{smeil}
% \end{itemize}



%%% Local Variables:
%%% mode: latex
%%% TeX-master: "../master"
%%% TeX-command-extra-options: "-enable-write18"
%%% End:
